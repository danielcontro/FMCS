\documentclass[11pt]{article}

\usepackage[T1]{fontenc}

\usepackage{amsmath}
\usepackage{amssymb}
\usepackage{amsfonts}
\usepackage{fontspec}
\usepackage{minted}
\usepackage{stmaryrd}
\usemintedstyle{emacs}

% \renewcommand{\theFancyVerbLine}{
%   \sffamily\textcolor[rgb]{0.5,0.5,0.5}{\scriptsize\arabic{FancyVerbLine}}}
\newcommand{\mintedOptions}{fontsize=\scriptsize}
\newcommand{\minline}[1]{{\footnotesize \mintinline{python}{#1}}}
\setmonofont{JetBrainsMono Nerd Font Mono}

\title{Verification of \\ General Reactivity of rank 1 formulas}
\author{Daniel Eduardo Contro - 2062899}
\date{}

\begin{document}

\maketitle

\section{Introduction}

The General Reactivity of rank 1 formulas, in short GR(1), are a special class
of formulas with the following shape:
\[
	(\Box\Diamond f_1 \wedge .. \wedge \Box\Diamond f_n) \to
	(\Box\Diamond g_1 \wedge .. \wedge \Box\Diamond g_m)
\]
or more generally
\[
	\left(\bigwedge_{i = 1}^n\Box\Diamond f_i \right) \to
	\left(\bigwedge_{j = 1}^m\Box\Diamond g_j \right)
\]
A system $TS$ satisfies a specification $\varphi$ if every infinite execution
satisfies $\varphi$ or in other words
\[
	TS \vDash \varphi \Leftrightarrow \forall\, \rho.\; \rho \vDash \varphi
\]
hence
\[
	TS \nvDash \varphi \Leftrightarrow \exists\, \rho.\; \rho \vDash \neg\varphi
\]
thus it will be searched a trace $\rho$ such that $\rho \vDash \neg\varphi$
to prove that $TS \nvDash \varphi$.
If such a trace doesn't exists then it must be the case that
$TS \vDash \varphi$.\\
In particular in the case of GR(1) formulas
\[
	\neg\varphi =
	\left(\bigwedge_{i = 1}^n\Box\Diamond f_i \right) \wedge
	\left(\bigvee_{j = 1}^m\Diamond\Box \neg g_j \right)
\]
\section{Correctness}

In this section it will be discussed the correctness of the algorithms
implemented for the verification of the satisfiability of GR(1) formulas by a
transition system $TS$ and for the construction of a counterexample execution
that invalidates the given GR(1) formula in $TS$.

\subsection{Specification satisfiability}

In order to prove that the transition system $TS$ satisfies the given GR(1)
property $\varphi$ it will be searched whether there exist executions which
trace $\rho$ satisfies $\neg\varphi$.
\[
	\neg\varphi = \left(\bigwedge_{i = 1}^n\Box\Diamond f_i \right) \wedge
	\left( \bigvee_{j = 1}^m\Diamond\Box \neg g_j \right)
\]
If such an execution exists then
$ \exists\, \rho.\ \rho \vDash \neg\varphi $
thus $TS \nvDash \varphi$, otherwise
$ \forall\, \rho.\ \rho \vDash \varphi $
hence $TS \vDash \varphi$.
\\ \\
First of all compute the reachable states of $TS$ using the
\minline{reach_from_region} function.

\begin{minted}[fontsize=\scriptsize]{python}
def check_explain_gr1_spec(
    spec: Spec,
) -> Optional[Tuple[bool, Optional[Counterexample]]]:
    parse_result = parse_gr1(spec)
    if parse_result == None:
        return None

    ts = pynusmv.glob.prop_database().master.bddFsm
    f_set, g_set = parse_result
    reach, reach_frontiers = reach_from_region(
        ts, ts.init, Method.POST, BDD.true())
    ...
\end{minted}

Then for each $g_j$ of the set $g$ of formulas compute the region of
reachable states satisfying
$ \left(\bigwedge \Box\Diamond f \right) \wedge
	\Box \neg g_j$
as follows:
\begin{itemize}
	\item compute the region of reachable states satisfying $\neg g_j$
	\item compute a fixpoint of the post function of the obtained region such
	      that for each state $s' \in$ \minline{not_gj_post_fixpoint} there
	      exist a state\\
	      $s \in$ \minline{not_gj_post_fixpoint} and an input $i$ such that
	      $s \overset{i}{\to} s' \in \llbracket React\rrbracket$.\\ \\
	      If \minline{not_gj_post_fixpoint} = $\emptyset$ it means
	      that there can't be a reachable state satisfying
	      $ \Box \neg g_j $
	      thus proceed with $g_{j+1}$; otherwise compute the recur regions in
	      \minline{not_gj_post_fixpoint} of the various $f_i$ satisfying
	      $ \left(\bigwedge_{1..i} \Box\Diamond f_i \right) \wedge
		      \Box \neg g_j $
	      using the \minline{meet_GF_f} function.
	      \begin{itemize}
		      \item If \minline{result} is not \minline{None} then there exists a
		            reachable state satisfying
		            $ \left(\bigwedge \Box\Diamond f \right) \wedge
			            \Box \neg g_j $ hence satisfying
		            $ \left(\bigwedge \Box\Diamond f \right) \wedge
			            \Diamond\Box \neg g_j $, thus build a counterexample.
		      \item Otherwise proceed with $g_{j+1}$
	      \end{itemize}

	\item If $\nexists\, g_j \in g$ such that the region of reachable states
	      satisfying \\
	      $ \left(\bigwedge \Box\Diamond f \right) \wedge
		      \left( \Diamond\Box \neg g_j \right)$
	      isn't empty then there's no execution which trace $\rho$ satisfies
	      $ \left(\bigwedge \Box\Diamond f \right) \wedge
		      \left( \bigvee \Diamond\Box \neg g \right) $
	      hence\\
	      $ \forall\, \rho. \rho \vDash \neg\left( \bigwedge \Box\Diamond f \right) \vee
		      \left( \bigwedge \Box\Diamond g \right) =
		      \left( \bigwedge \Box\Diamond f \right) \to
		      \left( \bigwedge \Box\Diamond g \right) = \varphi $\\
	      thus $TS \vDash \varphi$.
\end{itemize}

\begin{minted}[fontsize=\scriptsize]{python}
def check_explain_gr1_spec(
    spec: Spec,
) -> Optional[Tuple[bool, Optional[Counterexample]]]:
    ...
    for gj in iter(g_set):
        not_gj_reachable = reach & (-spec_to_bdd(ts, gj))
        not_gj_post_fixpoint = method_fixpoint(
            ts, Method.POST, not_gj_reachable)

        if not_gj_post_fixpoint.is_false(): continue

        result = meet_GF_f(ts, f_set, not_gj_post_fixpoint)
        if result:
            f_set_recurs, f_set_recur_loopable_region = result
            if not (result := get_counterexample(
                    ts, f_set_recurs, f_set_recur_loopable_region, reach_frontiers
                )):
                raise Exception("Counterexample not found")
            counterexample, states, _ = result
            return False, counterexample
    return True, None
\end{minted}

\subsubsection{Reachability}

Use \minline{_REACHES} for memoizaition and compute the reachable states with
BFS filtering each region transition with \minline{filter} to ensure that the
reach stays within the desired region.

\begin{minted}[fontsize=\scriptsize]{python}
_REACHES: Dict[Tuple[Method, Start, Filter], Tuple[Reach, Frontiers]] = {}

def reach_from_region(
    ts: BddFsm, region: BDD, method: Method, filter: Filter
) -> Tuple[Reach, Frontiers]:
    result = _REACHES.get((method, region, filter))
    if result: return result

    if method == Method.PRE: reach_method = ts.pre
    elif method == Method.POST: reach_method = ts.post

    reach = region & filter
    frontiers = [reach]
    while frontiers[-1].isnot_false():
        frontiers.append((reach_method(frontiers[-1]) - reach) & filter)
        reach = reach | frontiers[-1]
    _REACHES[(method, region, filter)] = (reach, frontiers)
    return reach, frontiers
\end{minted}

\subsubsection{Method fixpoint}

Repeatedly compute the conjunction between the \minline{current} region and
the \minline{next} one obtained applying the transition method to
\minline{current} until a fixpoint or an empty set is reached.

\begin{minted}[fontsize=\scriptsize]{python}
def method_fixpoint(ts: BddFsm, method: Method, region: BDD) -> BDD:
    if method == Method.POST: fixpoint_method = ts.post
    elif method == Method.PRE: fixpoint_method = ts.pre
    current = region
    while current.isnot_false():
        next = fixpoint_method(current) & current
        if current.equal(next):
            return current
        current = next
    return BDD.false()
\end{minted}

\subsubsection{$\bigwedge \Box\Diamond f$}

Compute whether $TS$ satisfies $\bigwedge \Box\Diamond f$ within the
\minline{reach} region. In order to do so for each $f_i \in f$.
\begin{itemize}
	\item compute the region within \minline{loopable_region} (where
	      $\bigwedge_{1..i-1} \Box\Diamond f_k$ is satisfied) satisfying
	      $\Box\Diamond f_i$; if it's empty then $\bigwedge \Box\Diamond f$
	      isn't satisfied, otherwise $\bigwedge_{1..i} \Box\Diamond f_k$ is
	      satisfied.
	\item update \minline{loopable_region} by computing the reach of the recur
	      region of $f_i$ filtering by its \minline{pre_reach} (states that can
	      reach the recur region of $f_i$ in at least one transition), the newly
	      computed one contains the states satisfying
	      $\bigwedge_{1..i} \Box\Diamond f_k$
\end{itemize}

\begin{minted}[fontsize=\scriptsize]{python}
def meet_GF_f(
    ts: BddFsm, f_set: Set[Spec], reach: PostReach
) -> Optional[Tuple[List[Recur], BDD]]:
    fi_recur = BDD.false()
    fi_recurs = []
    loopable_region = reach

    for fi in iter(f_set):
        result = GF_f(ts, fi, loopable_region, loopable_region)
        if not result: return None
        fi_recur, pre_reach = result
        fi_recurs.append(fi_recur)
        loopable_region, _ = reach_from_region(
            ts, fi_recur, Method.POST, pre_reach)
    return fi_recurs, loopable_region
\end{minted}

\subsubsection{$\Box\Diamond f$}

Repeatability check algorithm modified by adding a filter for ensuring that
\minline{f} is repeatable within the \minline{filter} region.

\begin{minted}[fontsize=\scriptsize]{python}
def GF_f(
    ts: BddFsm, f: Spec, region: BDD, filter: Filter
) -> Optional[Tuple[Recur, PreReach]]:
    recur = region & spec_to_bdd(ts, f) & filter
    while recur.isnot_false():
        pre_reach = BDD.false()
        new = ts.pre(recur) & filter
        while new.isnot_false():
            pre_reach = pre_reach | new
            if recur.entailed(pre_reach):
                return recur, pre_reach
            new = (ts.pre(new) - pre_reach) & filter
        recur = recur & pre_reach
    return None
\end{minted}

\subsection{Counterexample}

Once the existence of at least an execution which trace $\rho$ satisfies
$\neg\varphi$ has been proven build such a counterexample.
\begin{itemize}
	\item First compute a list of states looping and passing through the recur
	      regions and staying within the \minline{loopable_region} using the\\
	      \minline{get_loop_through_regions} function; each state in the loop
	      satisfies
	      $(\bigwedge \Box\Diamond f) \wedge (\bigvee \Diamond\Box \neg g)$.
	\item (If a state in the loop intersects the initial states then rotates the
	      loop to make it start from that state, for shortening the
	      counterexample)
	\item Then compute the shortest path (possibly empty) between the region of
	      initial states and the first state of the loop  using the
	      \minline{get_path} function and concatenate the obtained trace which
	      then satisfies
	      $\Diamond((\bigwedge \Box\Diamond f) \wedge (\bigvee \Diamond\Box \neg g))$
	      thus satisfying
	      $(\bigwedge \Box\Diamond f) \wedge (\bigvee \Diamond\Box \neg g)$.
	\item If the loaded transition system has inputs generate the inputs between
	      each state and its successor in the list of states computed and return
	      the built counterexample.
\end{itemize}

\begin{minted}[fontsize=\scriptsize]{python}
def get_counterexample(
    ts: BddFsm, recurs: List[Recur], loopable_region: BDD, reach_frontiers: Frontiers
) -> Optional[Tuple[Counterexample, Path, List[Inputs]]]:
    loop = get_loop_through_regions(
        ts, recurs, reach_frontiers, loopable_region)
    if loop == None: return None
    loop = rotate_loop(ts, loop)
    prefix = get_path(ts, ts.init, loop[0], BDD.true())
    if prefix == None: return None
    states = prefix[:-1] + loop
    inputs: List[Inputs] = []
    counterexample: List[Dict] = []

    has_inputs = len(ts.bddEnc.inputsVars) > 0

    for i in range(len(states) - 1):
        if not ts.post(states[i]).intersected(states[i + 1]):
            raise Exception("No transition between states", i, "and", i + 1)
        counterexample.append(states[i].get_str_values())
        if has_inputs:
            inputs.append(
                ts.pick_one_inputs(
                    ts.get_inputs_between_states(states[i], states[i + 1])
                )
            )
            counterexample.append(inputs[-1].get_str_values())
        else:
            counterexample.append({})
    counterexample.append(states[-1].get_str_values())
    return tuple(counterexample), states, inputs
\end{minted}

\subsubsection{Loop through regions}

Compute a list of states within \minline{filter}, that loops and intersects
all \minline{regions}.
\begin{itemize}
	\item Initialise \minline{start_region}, which is the region from which a
	      loop could start, as the states in the last region of
	      \minline{regions} within \minline{filter}, if it's empty then there's
	      no loop.
	\item Initialise \minline{waypoint_regions}, which are the regions that will
	      be used and updated searching the counterexample.
\end{itemize}
Until \minline{start_region} isn't empty:
\begin{enumerate}
	\item Pick one state between the "fastest" reachable states of \minline{ts}
	      inside \minline{start_region} and remove it from
	      \minline{start_region}
	\item Check if \minline{start} is reachable from \minline{start}, if not
	      repeat the previous steps from 1 for finding a different
	      \minline{start}
	\item Initialise \minline{loop}, which represents the actual loop, and
	      \minline{loop_waypoints}, which tracks the states of
	      \minline{waypoint_regions} visited in the loop.
	      \\ \\
	      At each iteration, while the number of \minline{loop_waypoints} is
	      greater than 0, tries computing the loop segment between the last
	      \minline{loop_waypoints} and \minline{waypoint_regions[i-1]}:
	      \begin{enumerate}
		      \item Assign \minline{current_region}, which is the region that will
		            be possibly reached by the loop in the current iteration, to
		            either the start or the first region of
		            \minline{waypoint_regions} not yet visited depending on
		            whether the loop must be closed at this iteration or not.
		      \item Compute the reach of the last waypoint visited.\\ \\
		            If \minline{current_region} or \minline{start} aren't in the
		            frontiers of the reach of the last waypoint then it means that
		            either the start of the loop or the next waypoint region
		            isn't reachable from the last waypoint thus remove it from
		            \minline{loop_waypoints} and repeat from \textbf{(a)} the
		            previous steps for finding a different waypoint for the
		            previous region.
		      \item Pick one state between the "fastest" reachable states from the
		            last loop waypoint that are in \minline{current_region} and
		            remove it from its \minline{waypoint_regions}
		      \item Build a path from the last loop waypoint to
		            \minline{next_state} and append it to \minline{loop} (removing
		            the first state since it's already included in loop)
		      \item If the number of \minline{regions} is the same as the number
		            of \minline{loop_waypoints} then the loop is completed thus
		            return it, else append \minline{next_state} to
		            \minline{loop_waypoints} and iterate looking for the loop
		            segment between \minline{next_state} and the next region in
		            \minline{waypoint_regions} not yet visited.
	      \end{enumerate}
	      If \minline{loop_waypoints} has no elements then there's no loop
	      passing through all \minline{regions} starting in \minline{start},
	      thus repeat from 1 to search for a different \minline{start}.
\end{enumerate}
otherwise there's no reachable loop passing through \minline{regions} within
\minline{filter}.

\begin{minted}[fontsize=\scriptsize]{python}
def get_loop_through_regions(
    ts: BddFsm, regions: List[BDD], reach_frontiers: Frontiers, filter: Filter
) -> Optional[Loop]:
    start_region = regions[-1] & filter
    if start_region.is_false(): return None

    waypoint_regions = regions
    while start_region.isnot_false():
        start_index = get_index_of(start_region, reach_frontiers)
        if start_index == None: return None
        start = ts.pick_one_state(start_region & reach_frontiers[start_index])
        start_region = start_region - start
        if not reach_from_region(
                ts, ts.post(start), Method.POST, filter
            )[0].intersected(start):
            continue

        loop = loop_waypoints = [start]
        while i := len(loop_waypoints):
            _, frontiers = reach_from_region(
                ts, loop_waypoints[-1], Method.POST, filter
            )
            if i == len(regions):
                current_region = start
            else:
                current_region = waypoint_regions[i - 1] & filter

            index = get_index_of(current_region, frontiers)
            start_index = get_index_of(start, frontiers)
            if index == None or start_index == None:
                waypoint_regions[i - 1] = regions[i - 1]
                loop_waypoints.pop()
                continue
            next_state = ts.pick_one_state(current_region & frontiers[index])
            waypoint_regions[i - 1] = waypoint_regions[i - 1] - next_state

            segment = get_path(
                ts, loop_waypoints[-1], next_state, filter, i == len(regions)
            )
            if segment == None: break

            loop = loop + segment[1:]
            if i == len(regions): return loop
            loop_waypoints.append(next_state)
    return None
\end{minted}

\subsubsection{Path}

Builds a path within \minline{filter} from \minline{start} to \minline{end},
ensuring at least one transition if the homonymous flag is \minline{True};
in the case where such a path doesn't exists returns \minline{None}.
\begin{enumerate}
	\item Filters out states in \minline{start} and \minline{end} that aren't
	      in \minline{filter}
	\item if \minline{first} and \minline{end} are intersected then keeps in
	      \minline{end} the intersection and if at least one transition is
	      required remembers to add a transition and shifts \minline{first} to
	      its post region filtered
	\item Initialise \minline{current_region} as the end and pick one state
	      from the region and put it in the path
	\item Compute the reach frontiers of \minline{first} and uses them for
	      building the path from \minline{end} backwards until \minline{first}
	\item Lastly if a transition has to be added picks a state from the pre
	      region of \minline{first} intersected with \minline{start} and then
	      returns the path reversed in order to be in the correct order.
\end{enumerate}

\begin{minted}[fontsize=\scriptsize]{python}
def get_path(
    ts: BddFsm, start: Start, end: End, filter: Filter, ensure_transition: bool = False
) -> Optional[Path]:
    add_ts = False
    first = start & filter
    end = end & filter
    if first.intersected(end):
        end = first & end
        if ensure_transition:
            add_ts = True
            first = ts.post(first) & filter

    current_region = end & filter
    path = [ts.pick_one_state(current_region)]
    first_reach, frontiers = reach_from_region(ts, first, Method.POST, filter)

    if not first_reach.intersected(end): return None

    if end_index := get_index_of(end, frontiers) == None: return None

    for i in range(end_index - 1, -1, -1):
        current_region = frontiers[i] & ts.pre(path[-1])
        path.append(ts.pick_one_state(current_region))

    if add_ts:
        current_region = start & ts.pre(first)
        path.append(ts.pick_one_state(current_region))

    return path[::-1]
\end{minted}

\section{Other aspects}

It has as well been implemented a recursive solution for computing
counterexamples using the \minline{recursive_get_loop_through_regions}
function.

\end{document}
